\documentclass[12pt,a4paper]{article}
\XeTeXlinebreaklocale "zh"
\XeTeXlinebreakskip = 0pt plus 1pt minus 0.1pt
\usepackage[top=1.25in,bottom=1.25in,left=1.25in,right=1in]{geometry}
\usepackage{float}
\usepackage{fontspec}
\newfontfamily\zhfont[BoldFont=Adobe Heiti Std]{Adobe Song Std}
\newfontfamily\zhpunctfont{Adobe Song Std}
\newfontfamily\consolas{Consolas}
\setmainfont{Times New Roman}
\setmonofont{Consolas}
\usepackage{indentfirst}
\usepackage{zhspacing}
\zhspacing

\usepackage[usenames,dvipsnames]{xcolor} % Required for custom colors

\usepackage{verbatim}
\usepackage{fancyhdr} % Required for custom headers
\usepackage{lastpage} % Required to determine the last page for the footer
\usepackage{extramarks} % Required for headers and footers
\usepackage{graphicx} % Required to insert imagesen
\usepackage{tabu} % for instruction encoding explaination
\usepackage{cases}

\usepackage{textcomp}

\usepackage[bottom]{footmisc} %让脚注在每一页重新编号和在章节、标题上加脚注

\usepackage[colorlinks,
            linkcolor=black,
            anchorcolor=black,
            citecolor=black
            ]{hyperref}

% Set up the header and footer
\pagestyle{fancy}
\lhead{\hmwkAuthorName} % Top left header
\chead{\hmwkClass\ (\hmwkClassInstructor): \hmwkTitle} % Top center head
\rhead{\today} % Top right header
\lfoot{\lastxmark} % Bottom left footer

\renewcommand\headrulewidth{0.4pt} % Size of the header rule
\renewcommand\footrulewidth{0.4pt} % Size of the footer rule

\linespread{1.1} % Line spacing

%----------------------------------------------------------------------------------------
%	DOCUMENT STRUCTURE COMMANDS
%	Skip this unless you know what you're doing
%----------------------------------------------------------------------------------------

\renewcommand{\contentsname}{目录}
\renewcommand{\today}{\number\year 年\number\month 月\number\day 日}

% Header and footer for when a page split occurs within a problem environment
\newcommand{\enterProblemHeader}[1]{
\nobreak\extramarks{#1}{#1接下页}\nobreak
\nobreak\extramarks{#1 (续)}{#1接下页}\nobreak
}

% Header and footer for when a page split occurs between problem environments
\newcommand{\exitProblemHeader}[1]{
\nobreak\extramarks{#1 (续)}{#1接下页}\nobreak
\nobreak\extramarks{#1}{}\nobreak
}

\setcounter{secnumdepth}{0} % Removes default section numbers
\newcounter{homeworkProblemCounter} % Creates a counter to keep track of the number of problems

\newcommand{\homeworkProblemName}{}
\newenvironment{homeworkProblem}[1][题目\arabic{homeworkProblemCounter}]{ % Makes a new environment called homeworkProblem which takes 1 argument (custom name) but the default is "Problem #"
\stepcounter{homeworkProblemCounter} % Increase counter for number of problems
\renewcommand{\homeworkProblemName}{#1} % Assign \homeworkProblemName the name of the problem
\section{\homeworkProblemName} % Make a section in the document with the custom problem count
\enterProblemHeader{\homeworkProblemName} % Header and footer within the environment
}{
\exitProblemHeader{\homeworkProblemName} % Header and footer after the environment
}

\newcommand{\problemAnswer}[1]{ % Defines the problem answer command with the content as the only argument
\noindent
{\Large\textbf{答:\\}}
\indent
}

\newcommand{\homeworkSectionName}{}
\newenvironment{homeworkSection}[1]{ % New environment for sections within homework problems, takes 1 argument - the name of the section
\renewcommand{\homeworkSectionName}{#1} % Assign \homeworkSectionName to the name of the section from the environment argument
\subsection{\homeworkSectionName} % Make a subsection with the custom name of the subsection
\enterProblemHeader{\homeworkProblemName\ [\homeworkSectionName]} % Header and footer within the environment
}{
\enterProblemHeader{\homeworkProblemName} % Header and footer after the environment
}




%----------------------------------------------------------------------------------------
%	CODE INCLUSION CONFIGURATION
%----------------------------------------------------------------------------------------

\usepackage{listings} % Required for insertion of code
\renewcommand{\lstlistingname}{代码}
\definecolor{MyDarkGreen}{rgb}{0.0,0.4,0.0} % This is the color used for comments
\lstloadlanguages{[x86masm]Assembler} % Load Assembly syntax for listings, for a list of other languages supported see: ftp://ftp.tex.ac.uk/tex-archive/macros/latex/contrib/listings/listings.pdf
\lstset{language={}, % Use Assembly in this example
        frame=single, % Single frame around code
        basicstyle=\small\consolas, % Use small true type font
        keywordstyle=[1]\color{Blue}\bf, % Assembly functions bold and blue
        keywordstyle=[2]\color{Purple}, % Assembly function arguments purple
        keywordstyle=[3]\color{Blue}\underbar, % Custom functions underlined and blue
        identifierstyle=, % Nothing special about identifiers
        commentstyle=\usefont{T1}{pcr}{m}{sl}\color{MyDarkGreen}\small, % Comments small dark green courier font
        stringstyle=\color{Purple}, % Strings are purple
        showstringspaces=false, % Don't put marks in string spaces
        tabsize=5, % 5 spaces per tab
        %
        % Put standard Assembly functions not included in the default language here
        morekeywords={},
        %
        % Put Assembly function parameters here
        morekeywords=[2]{ORG, AJMP, MOV, JNB, CLR, SETB, END, MOVX, DPTR, CJNE, JBC, SJMP, CPL, RETI, INC, DEC, SUBB, MOVC, JNZ, RRC},
        %
        % Put user defined functions here
        morekeywords=[3]{},
       	%
        morecomment=[l][\color{Blue}]{...}, % Line continuation (...) like blue comment
        numbers=left, % Line numbers on left
        firstnumber=1, % Line numbers start with line 1
        numberstyle=\tiny\color{Blue}, % Line numbers are blue and small
        stepnumber=5 % Line numbers go in steps of 5
}

% Creates a new command to include a Assembly script, the first parameter is the filename of the script (without .s), the second parameter is the caption
\newcommand{\asmscript}[2]{
\begin{itemize}
\item[]\lstinputlisting[caption=#2,label=#1]{#1.asm}
\end{itemize}
}

